\chapter{Schlussbetrachtung}
\label{ch:conclusion}
Die neue Version der \mapp zeigt, dass 3D-Echtzeit-Rendering auch auf schwächeren, integrierten Grafikeinheiten mit sehr detaillierten Modellen und 60 \acs{FPS} möglich ist. Die Unity-Engine bot als Software-Grundlage mit ihren \acs{RP}s und Beleuchtungseinstellungen eine große Auswahl an Anpassungsmöglichkeiten, um den Leistungsanspruch der Szene auf die Leistungsfähigkeit des genutzten Gerätes anzupassen. Die Ladevorgänge und das Speichermanagement der App konnten durch die zahlreichen neuen Ansätze und Optimierungen deutlich effizienter als in der Vorgängerversion gestaltet werden. Der aktuelle Zustand des Quellcodes zeigt jedoch noch Optimierungsbedarf. Beispielweise ist die Datenkapselung innerhalb der \name{DataLoader}- und \name{SceneState}-Klasse nicht vollständig ausgebaut. Auch kleinere Fehler treten noch in unregelmäßigen Abständen beim Abspielen der Animationen auf. Die Aufteilung der App in einzelne Szenen erleichterte das Ergänzen zusätzlicher Ansichten und  das Wiederverwenden bereits erstellter UI-Elemente. Die Implementierung des 3D-Echtzeit-Rendering von Fahrzeugmodellen und die Überarbeitung der verwendeten Technik innerhalb der \mapp wurden somit erfolgreich umgesetzt.
%